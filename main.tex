\documentclass[11pt,a4paper]{article}

\usepackage{siunitx}
\usepackage{mhchem}
\usepackage{multirow}

\usepackage[pdftex]{color,graphicx}
\pagestyle{plain}
\usepackage{geometry}
\newgeometry{margin=2.0cm}

\newcommand{\ts}{\textsuperscript}
\newcommand{\ic}{\texttt}
\newcommand{\todo}{TODO: \texttt}

\usepackage[backend=biber,style=authoryear,sorting=nyt,dashed=false]{biblatex}
\renewcommand*{\nameyeardelim}{\addcomma\space}
\addbibresource{references/references.bib} % note the .bib is required

%Wrong spellings!
%parameterization (unless part of someone else's work)
%parameterizing
%Paracon

\begin{document}

\newgeometry{margin=2.0cm, top=1.8cm}

\begin{center}
    \Large{\textbf{Monitoring Committee Report IV}}\\[0.1cm]
    \large{Mark Muetzelfeldt}\\
    \normalsize{2pm on Friday 7\ts{th} July 2017 in 2U13}\\[0.1cm]		
    \rule{\textwidth}{0.2mm}
    \textbf{Project Title: }Development of scale-awareness in the representation of
    convective cloud systems\\
    \textbf{Monitoring Committee: }Dr Omduth Coceal and  Dr Andrew Turner\\
    \textbf{Supervisors: }Prof. Robert Plant, Prof. Peter Clark, Dr Steve Woolnough \\
    and Dr Alison Stirling (Met Office CASE supervisor)\\
    \rule{\textwidth}{0.2mm}
\end{center}

\section{Project overview}
\label{sec:Project Overview}

%A statistical physics stochastic convection parametrization scheme such as \cite{plant2008stochastic} requires a characterization of the unresolved pro
One route to providing scale-awareness in a deep convection parametrization scheme is through the use of a statistical physics parametrization scheme \parencite{plant2008stochastic}. In such a scheme, the breakdown of the quasi-equilibrium assumption of \cite{arakawa1974interaction} is modelled by assuming that the number and strength of clouds in a given region can be represented by sampling from a probability density function (PDF). This requires suitable PDFs from which to sample. In \cite{plant2008stochastic}, the distributions were first derived from theory \parencite{craig2006fluctuations}, and then were tested against numerical simulations from a Cloud Resolving Model (\todo{combine acronym and citation} CRM) \parencite{cohen2006fluctuations}. The CRM simulates an atmosphere in Radiative-Convective Equilibrium (RCE), with prescribed cooling representing radiation, so that the dynamics of the cloud to cloud interactions can be tested without the complicating addition of radiative feedbacks. A stochastic convection scheme such as this provides scale-awareness in the soft grey zone (\SI{10}{}-\SI{100}{km}) by broadening the distribution of the PDFs as the resolution of the model increases, which provides higher sub-grid variability as the grid-scale decreases.

Organization of deep convection in the atmosphere can take many forms, such as squall lines, tropical cyclones, mesoscale clusters and larger scale features such as the Madden-Julian oscillation. In the interests of choosing a problem that is interesting, tractable and on which significant progress can be made within the time scale of a PhD, we are focusing on squall lines, and how these might be represented in a deep convection parametrization scheme that is scale-aware. The route that we are taking to do this is to ask the question: ``What effect does convective organization have on the overall cloud statistics of idealized RCE models?'' (See PhD Plan, Question 1). We are varying the organization through the use of systematically varying shear profiles, as has been done in other studies \parentite{RE2001, CC2006II}. What makes our approach novel is that in \cite{CC2006II}, they only look at three different profiles that are not varied systematically, and do not provide details of how the statistics change with the shear profiles. In \cite{RE2001}, they do not look at the statistics of the convection, instead focusing on equilibrium thermodynamic and momentum flux profiles \todo{what else?}. 

We have presented our results in a companion paper \parencite{muetz2017effects}, and we find that there are specific changes as the shear is increased, and these are identifiable across different analyses. I summarize three key results here. First, linearly scaled deep shear profiles taken from the Marshall Islands experiments \cite{yanai} can be used to stimulate organization, with low shear cases showing little sign of clustering beyond around \SI{15}{km}. High shear cases show signs of clustering at up to \SI{50}{km}, and significant repression of convection at larger spatial scales. An intermediate case is identified, where there is some clustering and repulsion, but it is less in magnitude than the more organized cases. Identifying this case is possible because of the the use of varying profiles and performing the analysis of spatial clustering on all of these profiles. Second, we find that there is a clear relationship between the strength of shear and the mass flux per cloud: as the shear increases, the number of clouds with low mass fluxes decreases and the maximum mass flux per cloud increases. Third, we observe that the variance of the mass flux distribution increases as the spatial scale of the analysis is increased, in a manner similar to that seen in \cite{plant2008stochastic}. The analysis of spatial scales also reinforces the finding that the mass flux per cloud increases as the shear increases \todo{does it?}.

This work goes someway towards answering the question posed above. It therefore provides a route to modifying a stochastic convection parametrization scheme so that it includes some of the effects of organization. It only goes part of the way though, the subsequent questions raised in my PhD Plan will provide the link between working how the statistics change with shear and how to modify the convection parametrization scheme. Further questions are raised in \cite{muetz2017effects}. 

\subsection{Background reading}

I have been reading key papers that discuss the mechanisms for formation of squall lines. From an observational perspective, \cite{houze1977} gives a detailed account of a squall line detected off the west coast of Africa. Reading such observational studies gives me an appreciation of the key properties of squall lines, such as the amount of precipitation caused by the convective plumes versus the amount caused by the trailing stratiform region.  It also serves as a useful reminder that the simulations of squall lines that I am generating using an idealized CRM are no more than crude representations of the real thing, no matter how high the resolution or detailed the microphysics.

In terms of mechanisms for the formation of squall lines, \cite{TMM} provides a good starting point, with its focus on an extremely simplified mathematical model for how these organized features are generated. \todo{more}

\cite{RKW1988} argues forcefully that low level shear is the key mechanism for the generation of optimal, long-lived squall lines. The low level shear is necessary to generate optimal squall lines because it has opposite vorticity to the vorticity of the propagating cold pools formed by convection. The proposed mechanism creates a long-lived feature through preferential creation of daughter convective cells on the upshear side of the existing cells, and thus individual cells are not hypothesized to be long-lived. Instead, the long-lived feature is the squall line itself, which is made from continually forming and decaying convective cells. This backs up certain observational studies, such as that by \cite{houze}. There are noticeable absences from this picture though: namely they do not link the strength of the cold pool to the microphysical processes that cause it, and they do not claim to provide a mechanism for how the squall line with a long spatial extent forms in the first place. Two of the authors re-appraised their original work several years later: they find that the evidence for their mechanism still supports the numerical studies and observations \todo{check this}.

One study that had a large influence on my thinking was \cite{RE2001}. Here they look at the specific mechanism suggested by \cite{RKW1988} in some detail, performing systematic experiments with different shear profiles. They find \todo{}. As noted in \ref{sec:Project Overview}, they do not perform the kind of statistical analysis that we have performed though.

\todo{Chris Holloway's recent paper}

\todo{Sengupta et al.}

\todo{Stochastic Parameterization}

\section{Completed work}

\subsection{Modelling of organized convection}
\label{sec:modelling_org}

% Work done for Quo Vadis
% High res runs

\subsection{Analysis}
\label{sec:analysis}

% Detecting clouds
% Counting clouds
% Org. metric
% Histograms

\subsection{Model configuration}
\label{sec:modelling_config}

\todo{more??}
% GW damping
% Init profile
% vn10.7 + why it's important and a good idea

\subsection{Modelling workflow}
\label{sec:modelling_workflow}

% Records
% Repeatability
% Reproducibility
% Provenance
% Archiving

% Mainly technical.
% logging
% multi-expt: see Fig...
% file conversion (plus compression)
% analysis built in

% Mention:
% omnium + scaffold_analysis, plus links to repos
% fcm + git

\subsection{Issues encountered}
\label{sec:issues}

% Domain wide oscillation + Fig
% poss causes:
% - cooling rate
% - init profile
% - resolution
% Lack of org. using LLS
% Energy balance

\section{Future work}

\subsection{Cell tracking}
\label{sec:cell_tracking}

In the companion paper we noted that we subjectively observed convective cells having longer lifetimes when the model was subjected to a shear profile. It would be beneficial to be able to quantify this relationship, because this provides a possible means for modifying a stochastic deep cumulus parametrization scheme. The lifetime of a convective cell in \cite{plant2008stochastic} is taken as being a constant. If we could determine how the lifetimes depended on the shear then we would have another route for modifying the scheme. To determine these lifetimes quantitatively, I would like to use a cell tracking procedure. In the department, Thorwald Stein and Juwon Kim have developed such a procedure in Python, so this would seem like a natural candidate for me to use. This also ties in with the research of \cite{plant2009statistical}, where they use cloud tracking to find statistical properties of the lifetimes of clouds. 

\subsection{\SI{1}{km} resolution modelling}
\label{sec:1km_res}

In the near term (2 months), I will be repeating the analysis that I have performed so far at \SI{2}{km} resolution on \SI{1}{km} resolution simulations. From previous work, I believe that the organized convective structures that are produced by the higher-resolution simulations will be more realistic, and that \SI{1}{km} resolution should resolve the deep cumulus convection that we are in interested in more adequately. The analysis code that I have written should work with minimal changes at this resolution, so I do not foresee many problems with running these simulations. I would like to improve the initial profiles of $\theta$ and water vapour before doing these runs however, as this will reduce the required spin-up time and therefore the overall run time of the simulations. It is worth noting that the storage requirements will go up by a factor of 4 for these simulations, and so the time (and memory) required to perform the analysis will also increase. I will be paying close attention to the qualitative behaviour of these runs, making sure that the results are consistent with the runs at \SI{2}{km}. I will also be able to do direct quantitative comparisons between the two resolutions, as the analysis will be almost identical.

\subsection{\SI{250}{m} resolution modelling}
\label{sec:250m_res}

Building on the exploratory runs at \SI{250}{m}, I would like to perform similar experiments to the ones carried out at \SI{2}{km} at these resolutions. As with the \SI{1}{km} runs, it will be highly desirable to have good initial profiles to reduce spin-up time for these runs. There are some changes that are necessary to run at these resolutions. These include the need to increase the vertical resolution, to better resolve the vertical structure of the updraughts and downdraughts. This will probably use a resolution profile from the SingV model, or a custom one used for other high-resolution studies developed by Kirsty Hanley. The timestep of the model will also have to decrease for stability reasons. 

The increased spatial and temporal resolution lead to increased computational requirements of the order of 200 times higher than the \SI{2}{km} runs (64 times for the spatial, 3 times for temporal). There will also be large increases in the size of the storage requirements for the output. Both of these will necessitate careful planning and picking of diagnostics from the model, as each extra field that is output will come with extra costs. It also means that the analysis that I run at lower-resolutions may not work out of the box, as it may consume too much memory or take prohibitively long to run. I am mindful of these constraints but see none of them as insurmountable.

\subsection{Running a stochastic convection parametrization scheme}

Before the next monitoring committee in December, I would like to have started running some tests using a stochastic parametrization scheme in an idealized RCE model. One of the ParaCon postdoctoral researchers, Jian-Feng Gu, has some experience of running the idealized RCE version of the UM, using the Gregory-Rowntree convection parametrization scheme \parencite{gregory1990mass}, so I will be talking with him to find out suitable settings and initial profiles to use for these runs. I will be using a modified version of the Plant-Craig scheme \parencite{plant2008stochastic}, with modifications to include some of the effects of shear.

\section{Training record}
\label{sec:Training record}

\subsection{Posters, presentations and conferences}

In February I took part in the department's annual set of presentations, \textit{Quo Vadis}, which gave me an opportunity to practise presenting my work in front of a large audience and to set out the direction in which my work was going. I presented initial results demonstrating that organized convection could be generated by the use of shear profiles. The key findings at that stage were that deep shear profiles, such as the ones used in \cite{cohen2006fluctuations}, could be used to stimulate organization, and that it was possible to detect a signal of the organization by generating a measure of the likelihood of finding clouds at a particular distance from each cloud. This measure is the same as the one used in the companion paper \cite{muetz2017effects} - Figure 5. The work at that stage did not vary the shear profiles systematically, nor did I perform any analysis of the variance of the mass flux. It did however use \SI{1}{km} resolution, which I subjectively think creates more realistic looking squall lines (see Section \ref{sec:1km_res})

I also attended a Met Office workshop for the Met Office Academic Partnership (MOAP) in February. MOAP is designed to build bridges between scientific institutions and the Met Office, so that suitable synergies between the Met Office and other institutions can be realized. The workshop included overviews of the scientific work at the Met Office, and typical career progression at the Met Office. We also saw talks from people from each of the institutions, giving us a feel for the depth and breadth of the work carried out by the partner institutions. Two talks stood out for me. The first was a talk by Coralia Cartis on optimizing climate models by tuning their parameters using techniques taken from numerical optimization. The second a talk by Steven Boeing on a numerical atmospheric model he has been developing called MPIC (\todo{what does this stand for?}), a model based on fully Lagrangian parcel tracking. The attendees also had a chance to present their own work as a poster. I presented the same work as I had talked about in \textit{Quo Vadis}.

I will also be attending the 3\ts{rd} ParaCon Plenary in Cambridge on the 3\ts{rd} and 4\ts{th} of July. The groups involved in ParaCon will give feedback on their progress so far. Of particular interest to me will be the work on high-resolution simulations by Todd Jones, and the work by Jian-Feng Gu on parametrized RCE, although I discuss their results with them semi-regularly so there will not be that much there that is new to me. It will also hopefully give me a chance to discuss my research with other people and get feedback from them on my direction of travel and ideas for how I can improve what I am doing.

On the 10\ts{th}-14\ts{th} of July I will be in Delft, attending the ``The Future of Cumulus Parametrization'' workshop. Judging by the agenda, as well as the main themes of the workshop, there will be a lot of exciting research there that is relevant to mine. The theme for day 2 is ``The organization of convection'', and day 4 is ``Stochastic aspects and the grey zone'', so I hope to see a lot of talks that I can take inspiration from. Picking out a couple that I am excited by: ``Radiative-convective equilibrium and the organisation of convection - An observational perspective'' by Christian Jakob, and ``Development of stochastic models of convective cloud populations'' by Samson Hagos both look like they will be interesting. I will also be presenting what I have done so far, in the form of a poster, so will be able to use this as a chance to meet some people doing similar research and get suggestions on what I am doing.

\subsection{Transferable skills}

I have attended three RRDPs this year. The most useful was ``Presentation skills'', where I had a chance to appraise how I was doing when presenting, as well as pointing out a few mistakes that I make when presenting. I also went to 2 days of a course on Data Assimilation (DA). We were taught some of the basics of DA, along with being given hands-on experience of using DA for a very simple model.

In the interests of keeping my programming skills up to scratch, I went on a course on software development. This went over some of the basics of using \texttt{git}, of Object-Oriented Programming (OOP) in Python, and of using unit testing frameworks such as \texttt{nose}. In some ways this was a useful refresher, but I felt that they were trying to train us to be `enterprise' software engineers, without tailoring the course to the typical things that we have to achieve as scientific programmers. Also, the course felt like they were teaching us Python from the perspective of a Java programmer, and seemed to eschew the `pythonic' way of doing things.

My duties as a member of the PostGraduate Research (PGR) forum have increased - I am now co-chair with Joshua Talib. We will be responsible for airing the complaints of our fellow researchers in the termly meetings, and organizing the introduction week and \textit{Quo Vadis} presentations. The forum gives us a chance to influence decisions on the running of the postgraduate research, as well as providing a window into the running of the department.

\printbibliography[title={References}]

\newpage
\section*{Appendix}

\begin{figure}[h]
    \centering
    \includegraphics[width=400px]{figs/{u-an388_graph}.png}
    \caption{}
    \label{fig:cylc_graph}
\end{figure} 

% Around 400 words.
\subsection*{Training record}
\subsubsection*{Year 1}

\begin{itemize}
  \item RRDP: Intermediate/Advanced \LaTeX\ (4/11/2015)
  \item RRDP: You and your supervisor (11/11/2015)
  \item RRDP: Quality assurance in research (18/11/2015)
  \item RRDP (equivalent): UM Training (16-18/12/2015)
  \item RRDP (equivalent): Preparing to teach: Introduction to teaching and Learning (26/1/2016)
  \item Preparing to teach: Marking and feedback (26/1/2016)
  \item Preparing to teach: Laboratory demonstrating and leading small groups (27/1/2016)
  \item MONC Training course (9-10/2/2016)
  \item RRDP (equivalent): Fairbrother Lecture ``A slippery situation: melting ice in Antarctica'' (4/5/2016)
  \item ECMWF Parametrization of subgrid physical processes (16-20/5/2016)
\end{itemize}

\subsubsection*{Year 2}

\begin{itemize}
  \item RRDP: Managing your research project (17/11/2016)
  \item RRDP: How to write a thesis (24/1/2017)
  \item SCENARIO Data Assimilation Course (14-15/2/2017)
  \item RRDP: Presentation skills (7/3/2017)
  \item Software Development for scientists (8/3/2017, 28-29/3/2017)
\end{itemize}

\subsection*{Talks and conferences attended}

\begin{itemize}
  \item Climate Change 2013: The physical science basis. Institute of Physics (2/2014)
  \item Dame Julia Slingo: Taking the planet into uncharted territory: What climate models can tell us about the future (9/2014)
  \item SCENARIO NERC DTP Conference (9/6/2015)
  \item Climate Change in the run-up to the Paris conference: what has Physics got to say? (6/11/2015)
  \item RMetS talk: The risk and vulnerability of Europe to severe convective storms (6/4/2016)
  \item ParaCon Plenary 1 in Reading (27-28/6/2016)
  \item RMetS debate: What will make the public and politicians take climate change more seriously? (5/10/2016)
  \item RMetS talks: Come Rain or Come Shine (19/10/2016)
  \item COP22 Marrakech: Remote participation (11/11/2016)
  \item ParaCon plenary 2 in Leeds (6-7/12/2016)
  \item RMetS talks: Chaos and Confidence in Weather Forecasting (14/12/2016)
  \item ParaCon plenary 3 in Cambridge (3-4/7/2017)
  \item The Future of Cumulus Parametrization, Delft University of Technology (10-14/7/2017)
\end{itemize}

\subsection*{Talks and conferences presented at}

\begin{itemize}
  \item Presentation: "Effects of Shear on Cloud Field Organization". \textit{Quo Vadis}, University of Reading (1/2/2017)
  \item Poster: "Effects of Shear on Cloud Field Organization". Met Office Academic Partnership (MOAP), Met Office, Exeter (22/2/2017)
  \item Poster: "Effects of Vertical Shear on Cloud Field Organization and Variability". The Future of Cumulus Parametrization, Delft University of Technology (10-14/7/2017)
\end{itemize}

\end{document}